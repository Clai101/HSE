\section{Вывод базовых формул}
\subsection{Монохроматические волны}
В первую очередь надо вспомнить что для ЭМ (электромагнитный) волн 
ур. Максвелла преобразуются в:
\begin{gather}
    \begin{matrix}
        \rot E = -\cfrac{1}{c} \cfrac{\partial H}{\partial t} & div H = 0, \\
        \rot H = \cfrac{1}{c} \cfrac{\partial E}{\partial t} & div E = 0.
    \end{matrix}
\end{gather} 
По определению:
\begin{eqnarray}
    H = \rot A; \ 
    E = -\cfrac{1}{c}\cfrac{\partial A}{\partial t} - \grad \phi .
    \label{eq:fil}
\end{eqnarray} 
Воспользуемся нормировеой Лоренца $\partial_i A^i = 0$, \ref{eq:fil} перейдет в:
\begin{eqnarray}
    H = \rot A; \ E = -\cfrac{1}{c}\cfrac{\partial A}{\partial t}.
\end{eqnarray} 
Учитывая запаздывание потенциала, в следствии чего $A\inner{t - \cfrac{x}{c}}$:
\begin{gather}
    \begin{matrix}
    E = -\cfrac{1}{c} A' ; \\ 
    H = \insqr{\nabla, A} = e_{ijk} \partial_j A^k = 
    e_{ijk} \cfrac{\partial \inner{t - x^m /c}}{\partial x^j}  
    \cfrac{\partial A^k}{\partial \inner{t - x^m /c}} =
    -\cfrac{1}{c}e_{ijk} \delta_{jm} \partial_{m} A^k = 
    -\cfrac{1}{c}\insqr{n, A'},
    \end{matrix}
\end{gather}
где $A'$ обозначает дифф. по $\inner{t - \cfrac{x}{c}}$. 
В итоге мы пришли к уже известному результату, что в 
монохроматических волнах:
\begin{eqnarray}
    H = \insqr{n, E}.
\end{eqnarray} 

\subsection{Запаздывающие потенциалы}
В дали от заряда потенциал равен:
\begin{equation}
    d\phi = \cfrac{d e \inner{t - \cfrac{R}{c}}}{R}
\end{equation}
Где $R$ - растоояния от зарядо до точки где измеряем значение потенциала. 
Интегрирования по зарядам дет нам:
\begin{equation}
    \phi = \int \cfrac{1}{R} \rho\inner{r', t - \cfrac{R}{c}} dV'
    \label{eq:fil_p1}
\end{equation}
В $R $ выражается как $R = r - r'$,  $r' -$ 
вектор от начала координат до заряда, $r -$ 
век. от н.к. до точки измерения. Выражение для 
векторного потенциала плучатся аналогично:
\begin{equation}
    A = \cfrac{1}{c} \int \cfrac{j \inner{r',t - R/c}}{R} dV' 
    \label{eq:fil_p2}
\end{equation}

\subsection{Приближение для запаздывающих потенциалов}
Будем считаь что мы измеряем поля на расстониях $r \gg r'$, 
Такое приближение более чем обоснованно при рассмотрении диполя 
(из его определения). И тогда $R = \abs{r' - r} \approx r - \inner{n,r'}$, 
в данном приближении \ref{eq:fil_p1} и \ref{eq:fil_p2} упростяться:
\begin{equation}
    \phi = \cfrac{1}{r} \int \rho\inner{r', t - \cfrac{r - \inner{n,r'}}{c}} dV'
\end{equation}
Может показаться что $r$ взнаменателе протеворечит тому, 
что написано выше но при разложении можно заметить что получится 
$\cfrac{1}{r - \inner{n,r'}} \approx \cfrac{1}{r} + \cancelto{0}{\cfrac{\inner{n,r'}}{r^2}} + ...$.
Аналогичным образом получим:
\begin{equation}
    A = \cfrac{1}{cr} \int j \inner{r',t - \cfrac{r - \inner{n,r'}}{c}} dV',
\end{equation}
обозначив $t' = t - \cfrac{r - \inner{n,r'}}{c}$, $r = R$,
 а также переходя к дискретному виду получим:
\begin{equation}
    A = \cfrac{1}{cR} \sum j \inner{r',t'} = \cfrac{1}{cR} \sum ev.
\end{equation}