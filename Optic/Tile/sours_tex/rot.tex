\section{Вращающийся диполь}
Заметим что задача обладает аксиальной симметрией, поэтому в полярных координата
\begin{eqnarray}
    x &=& r \cos \omega t\\
    y &=& r \sin \omega t
\end{eqnarray}

\begin{gather}
    d = e
    \begin{pmatrix}
        x \\ y
    \end{pmatrix}
    = e
    \begin{pmatrix}
        r\cos \omega t \\ r\sin \omega t
    \end{pmatrix}
    \implies 
    \ddot d =
    \begin{pmatrix}
        -r \omega \sin \omega t \\ r \omega \cos \omega t
    \end{pmatrix}
\end{gather}

Аналогично подставим в \ref{eq:J_ddotd}, получим:
\begin{eqnarray}
    \mathfrak{I} = \cfrac{\omega^2 r^2 e^2 }{4\pi c^3} \sin^2 \phi do
\end{eqnarray}

Поляризацию определим как 







