\documentclass[a4paper]{article}

%Пакеты для математических символов:
\usepackage{amsmath} % американское математическое сообщество.
\usepackage{amssymb} % миллион разных значков и готический, ажурный шрифты.
\usepackage{amscd} % диаграммы, графики.
\usepackage{amsthm} % окружения теорем, определений и тд.
\usepackage{physics} % основные физические символы
%\usepackage{latexsym} % треугольники и пьяная стрелка.

%пакеты для шрифтов:
%\usepackage{euscript} % прописной шрифт с завитушками.
\usepackage{MnSymbol} % Значеки доказательства
\usepackage{verbatim} % улучшенный шрифт "пишущей машинки".
%\usepackage{array} % более удобные таблицы.
%\usepackage{multirow} % мультистолбцы в таблицах.
%\usepackage{longtable} % таблицы на несколько страниц.
%\usepackage{latexsym}

\usepackage{etoolbox}
\usepackage{slashbox} %Разделениени текста \backslashbox{}{}
\usepackage{collectbox} % Добавляет коробочки, можно складывать туда текст)

%Пакеты для оформления:
\RequirePackage[center, medium]{titlesec}% Стиль секций и заголовков
%\usepackage[x11names]{xcolor} % 317 новых цветов для текста.
%\usepackage{multicol} % набор текста в несколько колонн.
\usepackage{amsmath} % Мат. структуры
\usepackage{graphicx} % расширенные возможности вставки стандартных картинок.
\usepackage{subcaption} % возможность вставлять картинки в строчку
%\usepackage{caption} % возможность подавить нумерацию у caption.
\usepackage{wrapfig} % вставка картинок и таблиц, обтекаемых текстом.
\usepackage{cancel} % значки для сокращения дробей, упрощения, стремления.
\usepackage{misccorr} % в заголовках появляется точка, но при ссылке на них ее нет.
%\usepackage{indentfirst} % отступ у первой строки раздела
%\usepackage{showkeys} % показывает label формул над их номером.
%\usepackage{fancyhdr} % удобное создание верхних и нижних колонтитулов.
%\usepackage{titlesec} % еще одно создание верхних и нижних колонтитулов
\usepackage{hyperref} %ссылки

%Пакеты шрифтов, кодировок. НЕ МЕНЯТЬ РАСПОЛОЖЕНИЕ.
\usepackage[utf8]{inputenc} % кодировка символов.
%\usepackage{mathtext} % позволяет использовать русские буквы в формулах. НЕСОВМЕСТИМО С tempora.
\usepackage[T1, T2A]{fontenc} % кодировка шрифта.
\usepackage[english, russian]{babel} % доступные языки.



%Отступы и поля:
%размеры страницы А4 11.7x8.3in
\textwidth=7.3in % ширина текста
\textheight=10in % высота текста
\oddsidemargin=-0.5in % левый отступ(базовый 1дюйм + значение)
\topmargin=-0.5in % отступ сверху до колонтитула(базовый 1дюйм + значение)


%Сокращения
%Скобочки
\newcommand{\inrad}[1]{\left( #1 \right)}
\newcommand{\inner}[1]{\left( #1 \right)}
\newcommand{\infig}[1]{\left\{ #1 \right\}}
\newcommand{\insqr}[1]{\left[ #1 \right]}
\newcommand{\ave}[1]{\left\langle #1 \right\rangle}


%% Красивые <= и >=
\renewcommand{\geq}{\geqslant}
\renewcommand{\leq}{\leqslant}

%%Значек выполнятся
\newcommand{\per}{\hookrightarrow}

%%Кванторы чиловых можеств
\newcommand{\com}{\mathbb{C}}
\newcommand{\re}{\mathbb{R}}
\newcommand{\nat}{\mathbb{N}}

%% Более привычные греческие буквы
\renewcommand{\phi}{\varphi}
\renewcommand{\epsilon}{\varepsilon}
\newcommand{\Epsilon}{\mathcal{E}}
\newcommand{\stp}{$\filledmedtriangleleft$}
\newcommand{\enp}{$\filledmedsquare$}
\makeatletter
\newcommand{\sqbox}{%
    \collectbox{%
        \@tempdima=\dimexpr\width-\totalheight\relax
        \ifdim\@tempdima<\z@
            \fbox{\hbox{\hspace{-.5\@tempdima}\BOXCONTENT\hspace{-.5\@tempdima}}}%
        \else
            \ht\collectedbox=\dimexpr\ht\collectedbox+.5\@tempdima\relax
            \dp\collectedbox=\dimexpr\dp\collectedbox+.5\@tempdima\relax
            \fbox{\BOXCONTENT}%
        \fi
    }%
}
\makeatother
\newcommand{\mergelines}[2]{
\begin{tabular}{llp{.5\textwidth}}
#1 \\ #2
\end{tabular}
}
\newcommand\tab[1][0.51cm]{\hspace*{#1}}
\newcommand\difh[2]{\frac{\partial #1}{\partial #2}}
\newcommand{\messageforpeople}[1]{HSE Faculty of Physics \ \ HSE Faculty of Physics HSE Faculty of Physics \ \ HSE Faculty of Physics HSE Faculty of Physics \ \ HSE Faculty of Physics HSE Faculty of Physics \ \ HSE Faculty of Physics HSE Faculty of Physics \ \ HSE Faculty of Physics HSE Faculty of Physics \ \ HSE Faculty of Physics HSE Faculty of Physics \ \ HSE Faculty of Physics HSE Faculty of Physics \ \ HSE Faculty of Physics }



\numberwithin{equation}{section}

\begin{document}


\tableofcontents
\newpagestyle{main}{
\setfootrule{0.4pt}
\setfoot{}{\thepage}{\sectiontitle № \thesection}}
\pagestyle{main}

\section{Задание}

\begin{gather}
    \begin{pmatrix}
        0 & 4 & 4 & 14\\ 
        12 & 20 & 0 & 0\\ 
        3 & 11 & 6 & 21
    \end{pmatrix}
    =
    \begin{pmatrix}
        0 & 4.0\\ 
        12.0 & 20.0\\ 
        3.0 & 11.0
    \end{pmatrix}
    \begin{pmatrix}
        1.0 & 0 & -1.66667 & -5.83333\\ 
        0 & 1.0 & 1.0 & 3.5
    \end{pmatrix} 
\end{gather}

\begin{gather}
    A^+
    =
    \begin{pmatrix}
        0.27911 & 0.43254\\ 
        0.43254 & 0.74048\\ 
        -0.03265 & 0.01958\\ 
        -0.11425 & 0.06855
    \end{pmatrix}
    \begin{pmatrix}
        -0.14308 & 0.12893 & -0.18239\\ 
        0.08019 & -0.0283 & 0.11321
    \end{pmatrix}
    \approx
\end{gather}

\begin{gather}
    \approx
    \begin{pmatrix}
        -0.00524967620000001 & 0.0237447703 & -0.0019390195\\ 
        -0.002508732 & 0.0348117982 & 0.00493877020000001\\ 
        0.0062416822 & -0.0047636785 & 0.0081716853\\ 
        0.0218439145 & -0.0166702175 & 0.028598603
    \end{pmatrix}
\end{gather}



\section{Задание}
\begin{gather}
    \begin{pmatrix} 
        6.0 & 12.0 & 13.0 & 0 \\ 
        12.0 & 8.0 & 20.0 & 3.0 \\ 
        0 & 16.0 & 6.0 & 6.0
    \end{pmatrix}
    =
    \begin{pmatrix} 
        1.0 & 0 & 1.41666666666666667 & 0.375 \\ 
        0 & 1.0 & 0.375 & -0.1875 \\ 
        0 & 0 & 0 & 0
    \end{pmatrix}
    = 
    \begin{pmatrix} 
        1 & 0 & \cfrac{17}{12} & \cfrac{3}{8}  \\ 
        0 & 1 & \cfrac{3}{8} & -\cfrac{3}{16}\\ 
        0 & 0 & 0 & 0
    \end{pmatrix}
\end{gather}

\begin{eqnarray}
    x + \cfrac{17-z}{12}=\cfrac{3}{8}\\
    y + \cfrac{3}{8}z=-\cfrac{3}{16}
\end{eqnarray}

\begin{eqnarray}
    x + \cfrac{17-z}{12}=\cfrac{3}{8}\\
    y + \cfrac{3}{8}z=-\cfrac{3}{16}
\end{eqnarray}

Наименьшее решение находим спомощью псевдобратной матрици
\begin{gather}
    \begin{pmatrix} 
        0.0872395833333333 \\ 
        0.29656862745098 \\ 
        -0.0464920343137255
    \end{pmatrix}
\end{gather}



\section{Задание}
Всего 6 условий, при этом максимум вторая проихводная, значит представим 
полиномом 5 степени.
\begin{equation}
    P(x) = y = a + bx + cx^2 + dx^3 + kx^4 + lx^5
\end{equation}
\begin{eqnarray}
    P(0) &=& 5 \\
    P(1) &=& 1 \\
    \partial_xP(0) &=& 5 \\
    \partial_xP(1) &=& 9 \\
    \partial_x^2P(0) &=& 4 \\
    \partial_x^2P(1) &=& 8
\end{eqnarray}
Нам повезло и сразу можем нати несколько коэфицентов:
\begin{eqnarray}
    P(0) &=& a = 5 \\
    \partial_xP(0) &=& b = 5 \\
    \partial_x^2P(0) &=& 2c = 4 \\
\end{eqnarray}

\begin{equation}
    P_3(x) = y = dx^3 + kx^4 + lx^5
\end{equation}
\begin{eqnarray}
    P_3(1) &=& -11 \\
    \partial_xP_3(1) &=& 0 \\
    \partial_x^2P_3(1) &=& 4
\end{eqnarray}

Для остльного составим матрицу

\begin{gather}
    \begin{pmatrix}
        x^3 & x^4 & x^5 \\
        3x^2 & 4x^3 & 5x^4 \\
        6x & 12x^2 & 20x^3 \\
    \end{pmatrix}
    \begin{pmatrix}
        d \\ k \\ l
    \end{pmatrix}
    =
    \begin{pmatrix}
        -11 \\ 0 \\ -4
    \end{pmatrix}
    \implies
    \begin{pmatrix}
        1 & 1 & 1 \\
        3 & 4 & 5 \\
        6 & 12 & 20 \\
    \end{pmatrix}
    \begin{pmatrix}
        d \\ k \\ l
    \end{pmatrix}
    =
    \begin{pmatrix}
        -11 \\ 0 \\ -4
    \end{pmatrix}
\end{gather}
\href{https://www.desmos.com/calculator/vsnlbomsq6}{Ссылка на график}

\section{Задание} 
В дз я доказывал что кривые безье касаютя точек нчала и конца 
при этом в этих токах совпадаеют 1 производные аппроксимационной кривой 
и прямой которую оппроксимируем.

Из данного очевидно что 2 точки уже мы знаем это 
$\Xi_1=\cfrac{A+D}{2}, \ \Xi_2\cfrac{B+C}{2}$. Дальше мы могли 
бы взять иеще одну точку (для Ы аппроксимации очевидно, что 
это должна быть точка с незадействованых ребер) и получит 
кривую безье 3 стпени, а 
мы знам, что это всегда полином степени $\lq 2$. 
В таком случае мы сразу получили овал по определению, так как функции
и ее 1 производная будут непрерывны в точках $\Xi_1, \Xi_2$ 
согласно изложенныему в 1 абзаце.

\begin{gather}
    \begin{pmatrix}
        x \\ y
    \end{pmatrix}
    = 
    \begin{pmatrix}
        x_1 \\ y_1
    \end{pmatrix}
    (1-t)^2
    +
    \begin{pmatrix}
        x_2 \\ y_2
    \end{pmatrix}
    3t^2(1-t)^2
    +
    \begin{pmatrix}
        x_3 \\ y_3
    \end{pmatrix}
    3t^2(1-t)
    +
    \begin{pmatrix}
        x_4 \\ y_4
    \end{pmatrix}
    t^3
\end{gather}
\href{https://www.desmos.com/calculator/g6zkwzxneq}{Ссылка на график}






\section{Задание}

Приблизим мнгочленом Чебышева:

\begin{equation}
    -ax^2 -bx - c + x^3 - 4 x^2 + 3 x + 4 
    = 
    8U_3\inner{\cfrac{x-3}{4}}
    =
    x^3 - 9 x^2 + 19 x - 3
\end{equation}

\begin{equation}
    -ax^2 -bx - c 
    =
    - 5 x^2 + 15 x - 7
\end{equation}

\begin{equation}
    P_2(x) = 5x^{2}-15x+7
\end{equation}

\href{https://www.desmos.com/calculator/3pv6pjtghp}{Ссылка на график}








\section{Задание}
Напрмер при $q = -1$
\begin{equation}
    2x^2+y^2\inner{1-4q} + zy\inner{2q+2} + z^2\inner{1-4q} \implies
    2x^2+5y^2 + 5z^2
\end{equation}
Тогда это единичный шар относительно 
\begin{equation}
    \mu\inner{x, y, z} 
    = \sqrt{\inner{\cfrac{x}{\sqrt{2}}}^2 + \inner{\cfrac{y}{\sqrt{5}}}^2 
    + \inner{\cfrac{z}{\sqrt{5}}}^2}
\end{equation}


Норма единичного вектора:
\begin{equation}
    \mu\inner{1, 1, 1} 
    = \sqrt{\inner{\cfrac{1}{\sqrt{2}}}^2 + \inner{\cfrac{1}{\sqrt{5}}}^2 
    + \inner{\cfrac{1}{\sqrt{5}}}^2} = \cfrac{3}{\sqrt{10}}
\end{equation}




\section{Задание}
\begin{gather}
    A^TA = 
    \begin{pmatrix}
        12 & 3 & 20\\
        10 & 12 & 8\\
        19 & 12 & 17
    \end{pmatrix}  
\end{gather}

\begin{equation}
    \lambda_3 = -4 \implies 0 \ \lambda_2 = 10,2 \implies 0 \ \lambda_1 = 39,3 
\end{equation}


\begin{gather}
    \begin{pmatrix}
        0.799154\\ 0.587255\\ 1
    \end{pmatrix}
\end{gather}

\begin{gather}
    \begin{pmatrix}
        0.799154 & 0 & 0\\ 
        0.587255 & 0 & 0\\ 
        1 & 0 & 0
    \end{pmatrix}
    \begin{pmatrix}
        39.3 & 0 & 0\\ 
        0 & 0 & 0\\ 
        0 & 0 & 0
    \end{pmatrix}
    \begin{pmatrix}
        0.799154 & 0.587255 & 1\\ 
        0 & 0 & 0\\ 
        0 & 0 & 0
    \end{pmatrix}
    =
    \begin{pmatrix}
        25.04 & 18.4 & 31.33 \\
        18.4 & 13.52 & 23.02 \\
        31.33 & 23.02 & 39.2
    \end{pmatrix}
\end{gather}










\end{document}