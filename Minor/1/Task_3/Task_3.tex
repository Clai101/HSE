\section{Задание}
Всего 6 условий, при этом максимум вторая проихводная, значит представим 
полиномом 5 степени.
\begin{equation}
    P(x) = y = a + bx + cx^2 + dx^3 + kx^4 + lx^5
\end{equation}
\begin{eqnarray}
    P(0) &=& 5 \\
    P(1) &=& 1 \\
    \partial_xP(0) &=& 5 \\
    \partial_xP(1) &=& 9 \\
    \partial_x^2P(0) &=& 4 \\
    \partial_x^2P(1) &=& 8
\end{eqnarray}
Нам повезло и сразу можем нати несколько коэфицентов:
\begin{eqnarray}
    P(0) &=& a = 5 \\
    \partial_xP(0) &=& b = 5 \\
    \partial_x^2P(0) &=& 2c = 4 \\
\end{eqnarray}

\begin{equation}
    P_3(x) = y = dx^3 + kx^4 + lx^5
\end{equation}
\begin{eqnarray}
    P_3(1) &=& -11 \\
    \partial_xP_3(1) &=& 0 \\
    \partial_x^2P_3(1) &=& 4
\end{eqnarray}

Для остльного составим матрицу

\begin{gather}
    \begin{pmatrix}
        x^3 & x^4 & x^5 \\
        3x^2 & 4x^3 & 5x^4 \\
        6x & 12x^2 & 20x^3 \\
    \end{pmatrix}
    \begin{pmatrix}
        d \\ k \\ l
    \end{pmatrix}
    =
    \begin{pmatrix}
        -11 \\ 0 \\ -4
    \end{pmatrix}
    \implies
    \begin{pmatrix}
        1 & 1 & 1 \\
        3 & 4 & 5 \\
        6 & 12 & 20 \\
    \end{pmatrix}
    \begin{pmatrix}
        d \\ k \\ l
    \end{pmatrix}
    =
    \begin{pmatrix}
        -11 \\ 0 \\ -4
    \end{pmatrix}
\end{gather}
\href{https://www.desmos.com/calculator/vsnlbomsq6}{Ссылка на график}
