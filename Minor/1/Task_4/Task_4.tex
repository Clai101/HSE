\section{Задание} 
В дз я доказывал что кривые безье касаютя точек нчала и конца 
при этом в этих токах совпадаеют 1 производные аппроксимационной кривой 
и прямой которую оппроксимируем.

Из данного очевидно что 2 точки уже мы знаем это 
$\Xi_1=\cfrac{A+D}{2}, \ \Xi_2\cfrac{B+C}{2}$. Дальше мы могли 
бы взять иеще одну точку (для Ы аппроксимации очевидно, что 
это должна быть точка с незадействованых ребер) и получит 
кривую безье 3 стпени, а 
мы знам, что это всегда полином степени $\lq 2$. 
В таком случае мы сразу получили овал по определению, так как функции
и ее 1 производная будут непрерывны в точках $\Xi_1, \Xi_2$ 
согласно изложенныему в 1 абзаце.

\begin{gather}
    \begin{pmatrix}
        x \\ y
    \end{pmatrix}
    = 
    \begin{pmatrix}
        x_1 \\ y_1
    \end{pmatrix}
    (1-t)^2
    +
    \begin{pmatrix}
        x_2 \\ y_2
    \end{pmatrix}
    3t^2(1-t)^2
    +
    \begin{pmatrix}
        x_3 \\ y_3
    \end{pmatrix}
    3t^2(1-t)
    +
    \begin{pmatrix}
        x_4 \\ y_4
    \end{pmatrix}
    t^3
\end{gather}
\href{https://www.desmos.com/calculator/g6zkwzxneq}{Ссылка на график}





