\section{Задание} 
Дз я доказывал что кривые безье касаютя точек нчала и конца 
при этом в этих токах совпадаеют 1 производные аппроксимационной кривой 
и прямой которую оппроксимируем.

Из данного очевидно что 2 точки уже мы знаем это 
$\Xi_1=\cfrac{A+D}{2}, \ \Xi_2\cfrac{B+C}{2}$. Дальше мы могли 
бы взять иеще одну точку (для Ы аппроксимации очевидно, что 
это должна быть точка с незадействованых ребер) и получит 
кривую безье 3 стпени, а 
мы знам, что это всегда полином степени $\lq 2$. 
В таком случае мы сразу получили овал по определению, так как функции
и ее 1 производная будут непрерывны в точках $\Xi_1, \Xi_2$ 
согласно изложенныему в 1 абзаце.

С другой соры можно взять еще точки например 
вершины. В этом никто не гарантирует выпуклости фигуры.
Поэтому двате рассмотрим 4 точки 
$\infig{x_{i,k}}, k = \infig{0, 1, 2, 3}$ 
проведем между ними отрезки, так чтобы мы получили ломаную. 
Давайте приблизим ее кривой Безье. Как уже упомяналось выполнятся: 
\begin{eqnarray}
    f_i\inner{0} &=& x_{0,i} \\
    f_i'\inner{0} &=& -x_{0,i} + x_{1,i} \\
    f_i\inner{1} &=& x_{3,i} \\
    f_i'\inner{1} &=& -x_{2,i} + x_{3,i}
\end{eqnarray}
Так как мы кривые Безье это полиномы то они не прерывны, 
а следовательно для них будет выполняться теорема Лагрнжа. 
И так мы знаем что на концах производная принемат некоторые 
значения, в силу непреравности если окажтся так что производные 
имеют разный знак (что соответствует выпуклой фигуре из четырех точек
(это можно заметить просто по графику)), то так как мы уже получим 
что $f'(t)$ - уже полином 2стпенени имет постоянную производную   







