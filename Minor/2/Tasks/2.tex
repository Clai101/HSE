\section{Задание}

Мне показалось логично округлять до целых, так как матрица остается не вырожденной
но при этом становится диагональной, из-за чего легко искать обратную. 
\begin{gather}
    A = 
    \begin{pmatrix}
        -5.0 & 0\\ 
        0 & -1.0
    \end{pmatrix}
    ;
    \
    \Delta A =
    \begin{pmatrix}
        0.03 & -0.14\\ 
        -0.06 & 0.04
    \end{pmatrix}
\end{gather}
\begin{gather}
    b = 
    \begin{pmatrix}
        -5.0\\ 
        -1.0
    \end{pmatrix};
    \
    \Delta b = 
    \begin{pmatrix}
        -0.18\\ 
        -0.08
    \end{pmatrix}
\end{gather}
\begin{gather}
    A^{-1} =
    \begin{pmatrix}
        -0.2 & 0\\ 
        0 & -1.0
    \end{pmatrix}
\end{gather}

Таким образом число обусловленности и погрешность $d$:
\begin{eqnarray}
    \kappa_1 \inner{A} = 5, & \kappa_2 \inner{A} = 5 \\
    \delta_1 b = 23.08, & \delta_2 b = 25.89
\end{eqnarray}

Я получил погрешность:
\begin{eqnarray}
    5.18 \leq &\delta_1 x& \leq 115.38 \\
    5.18 \leq &\delta_2 x& \leq 129.43
\end{eqnarray}

Предлагаю посчитаь точно и убедиться в этом, 
не буду пояснять поск решений просто приведу результат:
\begin{gather}
    x_{real} = \begin{pmatrix}
        1.08\\ 
        1.19
    \end{pmatrix},
    \
    x = 
    \begin{pmatrix}
        1.0\\ 
        1.0
    \end{pmatrix},
    \
    \Delta x = 
    \begin{pmatrix}
        0.08\\ 
        0.19
    \end{pmatrix}
\end{gather}

Тогда получим натоящюю погрешность:
\begin{equation}
    \delta_1 x = 7.46, \
    \delta_2 x = 6.84
\end{equation}

Действительно $\delta_1 x$ и $\delta_2 x$ лежет в найденных интервалах.


