\section{Эксперемент}
\subsection{Часть}
\begin{equation}
    2KI + Br_2 \to 2KBr + I_2
\end{equation}


\begin{equation}
    Br_2 + N_2OH \to NaBr + NaBrO_3 + H_2O
\end{equation}

\begin{eqnarray}
    Br + e^- &\to& Br^- \\
    Br - 5e^+ &\to& Br^{5+}
\end{eqnarray}

\begin{equation}
    3I_2 + 6NaOH \to 5NaI + NaI O_3 + 3H_2O
\end{equation}


Происходит обесчвечивание так как в ходе рекции из $Br_2$ образуется 
бесцветное вещество.

\begin{equation}
    Na_2S + Br_2 \to 2NaBr + S\downarrow
\end{equation}

\subsection{Часть}
\begin{equation}
    2Fe + 3Br_2 \to 2FeBr_3
\end{equation}

\begin{equation}
    FeBr_3 + K_4\insqr{Fe\inner{CN}_6} \to 
    KFe\insqr{Fe\insqr{CN}_6} + 3KBr
\end{equation}
Свидетельством ионов железа являтся характерный синй цвет, что 
является вдетельством наличия сини в состав которой входят ионы железа.
\subsection{Часть}
\begin{eqnarray}
    Na_2S + Br_2 &\to& 2NaBr + S\downarrow \\
    Na_2S + I_2 &\to& 2NaI + S\downarrow
\end{eqnarray}
Выпадает желтовато-белый осадок.



