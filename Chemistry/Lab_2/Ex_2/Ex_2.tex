\section{Эксперимент}

\tab Правило Бертоли в каждой реакции можно 
найти такое соединение которое не растоворяется 
в воде тогда все реакции сотоятся:
\begin{eqnarray} 
 ZnCl_2 + Na_2S      &\rightarrow&  ZnS\downarrow + 2NaCl     \\
 CuSO_4 + Na_2S      &\rightarrow&  CuS\downarrow + Na_2SO_4  \\
 Pb(NO3)_2 + Na_2S   &\rightarrow&  PbS\downarrow + 2NaNO_3   \\
 MnCl_2 + Na_2S      &\rightarrow&  MnS\downarrow + 2NaCl 
\end{eqnarray} 

Все выше перечисленные реакции называются реакциями
замещения.

\begin{center}
    \begin{tabular}{l||l||l}
        Compound & Color & Transparency \\ \hline \hline
        $ZnS$ & Белый &     Непрозрачный \\
        $CuS$ & Черный &    Непрозрачный \\
        $PbS$ & Серый  &    Непрозрачный \\
        $MnS$ & Черный &    Непрозрачный
    \end{tabular}
\end{center}

Прибавим соляную кислоту:

\begin{eqnarray} 
    ZnS + 2NaCl + 2HCl      &\rightarrow& ZnCl2\downarrow + Na2S + H2O \\
    CuS + Na2SO4 + 4HCl     &\rightarrow& CuSO4\downarrow + 2NaCl + H2S \\
    PbS + 2NaNO3 + 6HCl     &\rightarrow& PbCl2\downarrow + 2NaNO3 + H2S \\
    MnS + 2NaCl + 2HCl      &\rightarrow& MnCl2\downarrow + Na2S + H2O
\end{eqnarray}

\begin{center}
    \begin{tabular}{l||l||l}
        Compound & Color & Transparency \\ \hline \hline 
        $ZnCl_2$ & Белый              &     Непрозрачный \\
        $CuSO_4$ & Черный             &     Непрозрачный \\
        $Pb(NO_3)_2$ & Черный          &    Непрозрачный \\
        $MnCl_2$ & Бледный бело-желтый &    Непрозрачный 
    \end{tabular}
\end{center}