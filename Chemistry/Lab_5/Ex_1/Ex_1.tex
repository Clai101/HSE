\section{Эксперемент}

\begin{equation} 
    (NH_4)_2S_2O_8 + KI \xrightarrow{} 
    I_2 + K_2SO_4 + (NH_4)_2SO_4 
\end{equation} 

Лимитирующей реакцией в нашем случае будет 

\begin{equation} 
  S_2O_8^{2 -} + I^{ -} \xrightarrow{} S_2O_8I^{3 -} 
\end{equation} 

Из-за большмх энерго затрат времени на абсорбцию энергии
из окружающей среды, в следствии чего она (реакция)
требутет ниибольшее количество времени.

\begin{equation} 
 \mu = \cfrac{\insqr{S_2O_3^{2 -}}}{2\insqr{S_2O_8^{2 -}}} 
\end{equation} 

Благодаря $2S_2O_3^{2-} + I_2 \xrightarrow{} 2I^{-} + S_2O_3^{2-}$ 
и принебрегая ипарнеиями и любой другой потерей конценрации, 
$\insqr{I^{ -}} = const$. 

\begin{equation} 
  \dif{\insqr{S_2O_8^{2 -}}}{t} = k \insqr{I^{ -}}_0 \cdot 
  \insqr{S_2O_8^{2 - }} \implies 
  \int_{\insqr{S_2O_8^{2 -}}_0}^{\insqr{S_2O_8^{2 -}}_\tau} 
  \cfrac{d\insqr{S_2O_8^{2 -}}}{\insqr{S_2O_8^{2 -}}} =
  \ln \inner{\cfrac{\insqr{S_2O_8^{2 -}}_\tau}{\insqr{S_2O_8^{2 -}}}}
  = k \insqr{I^{ -}}_0 \tau 
\end{equation} 

\begin{equation} 
 \mu = \cfrac{\insqr{S_2O_3^{2 -}}}{2\insqr{S_2O_8^{2 -}}} = 
 \cfrac{0.2}{2\cdot 0.5} = 0.2
\end{equation} 

\begin{equation} 
 \insqr{I^ - } =  \cfrac{1}{8(7 + n)}
\end{equation} 

\begin{equation} 
 k' = \inner{0.00275, \ 0.00603, \ 0.00769, \ 0.02479}
\end{equation} 

\begin{equation} 
  k = \inner{0.242, \ 0.482, \ 0.554, \ 1.587}
\end{equation} 

\begin{equation} 
 \ave{k} = 0.71625 
\end{equation} 

