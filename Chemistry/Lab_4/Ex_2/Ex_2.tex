\section{Эксперемент}

\begin{equation}
    m_{Cu} = \cfrac{M}{N_a} = 9.5\cdot10^{-25} g
\end{equation}
\begin{equation}
    \mathfrak{I} = \difh{q}{t}
\end{equation}
\begin{equation}
    \partial m = \cfrac{\partial qm_{Cu}e}{2}
\end{equation}
\begin{equation}
    \Delta m = \cfrac{m_{Cu}}{2e}\int_{0}^{\tau}  \mathfrak{I}\inner{t} dt 
\end{equation}

Нам удалось зафиксировать силу тока то формула:
\begin{equation}
    \Delta m = \cfrac{\mathfrak{I}tm_{Cu}}{2e}
\end{equation}
\begin{center}
    \begin{tabular}{l||l}
        Было & Сало \\
        30.1608 g & 30.1811g
     \end{tabular}
\end{center}

\begin{equation}
    \Delta m = 0.038 g
\end{equation}





